\documentclass[12pt,a4paper]{article}

% ============ PACKAGES ============
\usepackage[utf8]{inputenc}
\usepackage[T1]{fontenc}
\usepackage{geometry}
\usepackage{graphicx}
\usepackage{titlesec}
\usepackage{fancyhdr}
\usepackage{hyperref}
\usepackage{enumitem}
\usepackage{booktabs}
\usepackage{tabularx}
\usepackage{caption}
\usepackage{subcaption}
\usepackage{float}
\usepackage{parskip}
\usepackage{array}

% ============ PAGE SETUP ============
\geometry{margin=2.5cm, top=3cm, bottom=3cm}

% ============ HYPERREF ============
\hypersetup{
    colorlinks=false,
    pdfborder={0 0 0},
    pdftitle={QuickBite - HCI Design Report},
    pdfauthor={Y.Niranjan},
    bookmarks=true
}

% ============ SECTION FORMATTING ============
\titleformat{\section}
    {\Large\bfseries}{\thesection.}{8pt}{}
\titleformat{\subsection}
    {\large\bfseries}{\thesubsection}{8pt}{}
\titleformat{\subsubsection}
    {\normalsize\bfseries}{\thesubsubsection}{6pt}{}

% ============ HEADER / FOOTER ============
\pagestyle{fancy}
\fancyhf{}
\fancyhead[L]{\small QuickBite -- HCI Design Report}
\fancyhead[R]{\small Y.Niranjan | CS23B1076}
\fancyfoot[C]{\small \thepage}
\renewcommand{\headrulewidth}{0.4pt}

% ============ DOCUMENT ============
\begin{document}

% ============ TITLE PAGE ============
\begin{titlepage}
    \begin{center}
        \vspace*{2cm}

        {\fontsize{36}{40}\selectfont\bfseries QuickBite}\\[8pt]
        {\Large Online Food Delivery Application}

        \vspace{2cm}

        {\LARGE\bfseries HCI Design Report}\\[6pt]
        {\large Interactive Design Layout \& Scheme}\\[4pt]
        {\normalsize Mid-Semester Design Activity}

        \vspace{3cm}

        \begin{tabular}{rl}
            \textbf{Name:} & Y.Niranjan \\[6pt]
            \textbf{Roll No:} & CS23B1076 \\[6pt]
            \textbf{Course:} & Human-Computer Interaction (HCI) \\[6pt]
            \textbf{Assignment:} & Mid-Semester Design Activity \\[6pt]
            \textbf{Domain:} & Online Food Delivery Application \\[6pt]
            \textbf{Date:} & February 28, 2026 \\
        \end{tabular}

        \vfill

        {\small 12 Interactive Screens \textbullet{} Complete Prototype \textbullet{} HTML/CSS/JS}
    \end{center}
\end{titlepage}

% ============ TABLE OF CONTENTS ============
\newpage
\tableofcontents
\newpage

% ===================================================================
\section{Introduction}

\textbf{QuickBite} is a food delivery mobile application designed to provide a simple and intuitive food ordering experience. The design covers the complete user journey from onboarding to order tracking, with all decisions grounded in HCI principles.

\begin{itemize}
    \item 12 core screens covering the full ordering flow
    \item Built as an interactive HTML/CSS/JS prototype
    \item Mobile-first layout (390 x 844px)
    \item Typography: Poppins (Google Fonts)
    \item Icons: Font Awesome 6
\end{itemize}

\subsection{Screens Overview}

\begin{table}[H]
\centering
\small
\begin{tabularx}{\textwidth}{c l X}
\toprule
\textbf{\#} & \textbf{Screen} & \textbf{Purpose} \\
\midrule
1 & Splash Screen & Brand identity and loading indicator \\
2 & Onboarding (3 slides) & Feature introduction (Browse, Order, Enjoy) \\
3 & Login / Sign Up & Phone OTP authentication with social login options \\
4 & Home Dashboard & Main hub with categories and restaurants \\
5 & Search Screen & Discovery with filters, sort, trending \\
6 & Restaurant Listing & Menu browsing, customization, add to cart \\
7 & Cart Screen & Order review, coupon, bill breakdown, checkout \\
8 & Order Confirmation & Success screen with order details \\
9 & Order Tracking & Real-time status, map, delivery partner info \\
10 & Order History & Past orders, reorder, rate and review \\
11 & Profile / Account & Settings, addresses, payments \\
12 & Help \& Support & FAQ, live chat, issue reporting \\
\bottomrule
\end{tabularx}
\caption{Complete list of designed screens}
\end{table}

\subsection{Color Palette}

\begin{table}[H]
\centering
\small
\begin{tabular}{l l l l}
\toprule
\textbf{Color} & \textbf{Hex Code} & \textbf{Role} & \textbf{Usage} \\
\midrule
Orange & \texttt{\#FF6B35} & Primary & Buttons, CTAs, branding \\
Red & \texttt{\#D63031} & Secondary & Errors, offers, non-veg indicator \\
White & \texttt{\#FFFFFF} & Background & Screen backgrounds, cards \\
Light Gray & \texttt{\#F5F5F5} & Surface & Input fields, sections \\
Charcoal & \texttt{\#2D3436} & Text (dark) & Headings \\
Gray & \texttt{\#636E72} & Text (body) & Descriptions, meta info \\
Green & \texttt{\#00B894} & Accent & Success states, ratings, veg indicator \\
\bottomrule
\end{tabular}
\caption{Color palette}
\end{table}

% ===================================================================
\newpage
\section{80-20 Rule (Pareto Principle)}

The Pareto Principle states that roughly 80\% of effects come from 20\% of causes. In QuickBite, the critical 20\% of features receive the most design attention.

\subsection{Critical 20\% (Primary Focus)}
\begin{itemize}
    \item \textbf{Browse Food (Home):} Categories, restaurant cards, and promotional banners placed front-and-center
    \item \textbf{Search:} Prominent search bar with filters, trending items, and recent searches
    \item \textbf{Menu \& Add to Cart (Restaurant):} Clear menu items with one-tap add and customization popup
    \item \textbf{Cart \& Checkout:} Price breakdown, coupon field, and prominent ``Place Order'' button
    \item \textbf{Order Tracking:} Step-by-step progress indicator, ETA, map, and delivery partner info
\end{itemize}

\subsection{Supporting 80\% (Secondary)}
\begin{itemize}
    \item Profile management, order history, help \& support, notifications
    \item Accessed through bottom navigation or nested menus
    \item Organized but do not clutter the core ordering flow
\end{itemize}

The bottom navigation prioritizes core actions: Home, Search, Cart, Orders, Profile. The ordering flow (Browse $\to$ Select $\to$ Cart $\to$ Order $\to$ Track) is always within 1--2 taps.

% ===================================================================
\newpage
\section{Shneiderman's 8 Golden Rules}

\subsection{Rule 1: Strive for Consistency}
\begin{itemize}
    \item Same button styles, card patterns, corner radius, and color scheme across all 12 screens
    \item Bottom navigation maintains identical positioning and styling everywhere
    \item Poppins font used consistently for all text
\end{itemize}

\subsection{Rule 2: Seek Universal Usability / Enable Shortcuts}
\begin{itemize}
    \item ``Reorder'' button on past orders for quick repeat ordering
    \item Saved addresses (Home, Work) for one-tap delivery selection
    \item Recent searches displayed for quick re-search
    \item Keyboard shortcuts (1--0 keys) in the prototype for direct screen access
\end{itemize}

\subsection{Rule 3: Offer Informative Feedback}
\begin{itemize}
    \item Coupon applied shows green confirmation message
    \item Order placement shows a confirmation screen with confetti animation
    \item Adding items updates the cart badge with a scale animation
    \item Order tracking provides step-by-step updates with timestamps
    \item OTP input auto-advances between fields
\end{itemize}

\subsection{Rule 4: Design Dialogs to Yield Closure}
\begin{itemize}
    \item Ordering flow concludes with a dedicated confirmation screen
    \item Shows Order ID, restaurant name, estimated time, and total paid
    \item Green checkmark with animation provides clear sense of completion
\end{itemize}

\subsection{Rule 5: Prevent Errors / Simple Error Handling}
\begin{itemize}
    \item Phone number input validates 10-digit format in real time
    \item Invalid input shows clear error messages
    \item OTP boxes auto-advance to prevent misentry
    \item Full bill breakdown shown before placing order
\end{itemize}

\subsection{Rule 6: Permit Easy Reversal of Actions}
\begin{itemize}
    \item Cart items have +/- quantity controls and remove option
    \item Delivery address changeable via ``Change'' button
    \item Orders cancellable from the tracking screen
    \item Back buttons on every sub-screen
\end{itemize}

\subsection{Rule 7: Keep Users in Control}
\begin{itemize}
    \item Veg/Non-Veg filter toggles on search and restaurant screens
    \item Sort options (Rating, Cost, Time)
    \item Menu category navigation pills
    \item Customizable order options (size, toppings, special instructions)
    \item Notification preferences toggle in profile
\end{itemize}

\subsection{Rule 8: Reduce Short-term Memory Load}
\begin{itemize}
    \item Cart badge on bottom nav always shows item count
    \item Floating cart summary bar on restaurant page shows items and total
    \item Progress indicator on tracking screen shows current status at a glance
    \item Price breakdown in cart eliminates the need to mentally calculate totals
    \item Food categories use emoji icons for instant recognition
\end{itemize}

% ===================================================================
\newpage
\section{Nielsen's 10 Usability Heuristics}

\subsection{H1: Visibility of System Status}
\begin{itemize}
    \item Order tracking shows a 4-step progress indicator (Order Placed $\to$ Preparing $\to$ Out for Delivery $\to$ Delivered) with timestamps and ETA
    \item Splash screen has loading animation
    \item Banner carousel shows progress dots
    \item Cart badge updates in real time
\end{itemize}

\subsection{H2: Match Between System and Real World}
\begin{itemize}
    \item Food emoji for categories (pizza, biryani, burger, etc.)
    \item Star ratings for restaurants
    \item Map visualization for delivery tracking
    \item Natural language prompts (``What's on your mind?'')
    \item Standard Veg (green dot) / Non-Veg (red triangle) symbols
\end{itemize}

\subsection{H3: User Control and Freedom}
\begin{itemize}
    \item Back button on every screen
    \item Skip button on onboarding
    \item Cancel order from tracking screen
    \item Edit cart items and change address
    \item Close button on all popups
\end{itemize}

\subsection{H4: Consistency and Standards}
\begin{itemize}
    \item Bottom tab navigation (standard mobile pattern)
    \item Search bar at top of relevant screens
    \item Card-based layouts throughout
    \item Industry-standard Font Awesome icons
\end{itemize}

\subsection{H5: Error Prevention}
\begin{itemize}
    \item Input validation prevents invalid phone numbers before submission
    \item Auto-advancing OTP boxes reduce entry errors
    \item Full bill summary shown before checkout
    \item Coupon field provides instant validity feedback
\end{itemize}

\subsection{H6: Recognition Rather Than Recall}
\begin{itemize}
    \item Visual food categories with emoji icons
    \item Recent searches displayed on search screen
    \item Restaurant cards show key info (rating, time, distance) at a glance
    \item Order history shows past orders for easy reordering
    \item Saved addresses labeled (Home, Work)
\end{itemize}

\subsection{H7: Flexibility and Efficiency of Use}
\begin{itemize}
    \item Multiple paths to find food: browse categories, search, explore recommendations, or reorder
    \item Filters and sort for advanced users
    \item Quick reorder bypasses the entire browse-to-cart flow
\end{itemize}

\subsection{H8: Aesthetic and Minimalist Design}
\begin{itemize}
    \item Clean white backgrounds with generous whitespace
    \item Each screen focuses on its primary task
    \item Information hierarchy: large headings, supporting details, then actions
    \item Consistent 8px grid spacing
\end{itemize}

\subsection{H9: Help Users Recognize, Diagnose, and Recover from Errors}
\begin{itemize}
    \item Phone validation error states ``Please enter a valid 10-digit phone number''
    \item Specific error messages for all failure states
    \item Clear retry options for failed operations
\end{itemize}

\subsection{H10: Help and Documentation}
\begin{itemize}
    \item Dedicated Help \& Support screen with searchable FAQ accordion
    \item Live chat, call, and email support options
    \item Issue reporting form with order dropdown
    \item Onboarding slides introduce key features to new users
\end{itemize}

% ===================================================================
\newpage
\section{Tesler's Law (Conservation of Complexity)}

Tesler's Law states that every application has inherent complexity that cannot be removed --- it can only be moved. Good design moves complexity from the user to the system.

\begin{itemize}
    \item \textbf{Smart Defaults:} Default delivery address auto-selected (Home). Regular size pre-selected in customization popup. Default payment method remembered.
    \item \textbf{Step-by-step Checkout:} The cart screen breaks checkout into distinct sections (items $\to$ coupon $\to$ address $\to$ bill $\to$ place order) instead of showing everything at once.
    \item \textbf{Auto-computation:} Bill summary auto-calculates item total, delivery fee, taxes, discounts, and grand total. Users never need to do math.
    \item \textbf{Customization in Modal:} Item customization (size, toppings, instructions) is contained in a popup instead of cluttering the menu list.
    \item \textbf{Filter/Sort Defaults:} Filtering and sorting are available but optional --- the default view works well for casual users.
\end{itemize}

% ===================================================================
\section{Serial Position Effect}

People remember the first (primacy) and last (recency) items in a series best.

\subsection{Primacy Effect}
\begin{itemize}
    \item ``Home'' placed first (leftmost) in bottom navigation --- the most-used screen
    \item Promotional banners and offers appear first on the Home screen
    \item ``Recommended'' category appears first in the restaurant menu
\end{itemize}

\subsection{Recency Effect}
\begin{itemize}
    \item ``Profile'' placed last (rightmost) in bottom navigation
    \item ``Place Order'' button fixed at the bottom of the Cart screen --- the last visible element is the key action
    \item Floating cart bar stays at the bottom of the Restaurant page
\end{itemize}

\subsection{Bottom Navigation Layout}
\begin{center}
    \textbf{Home} $|$ Search $|$ \textbf{Cart} $|$ Orders $|$ \textbf{Profile}
\end{center}
Home (primacy) and Profile (recency) bookend the navigation. Cart holds the center position with a badge for visual anchoring.

% ===================================================================
\newpage
\section{Learnability (Law of Learning)}

The time to complete a task decreases with practice. QuickBite is designed to be instantly learnable.

\begin{itemize}
    \item \textbf{Familiar Patterns:} Tab navigation, card-based content, search with filters, cart with quantity controls, map-based tracking --- all patterns users already know.
    \item \textbf{Onboarding:} Three slides introduce core features: Browse $\to$ Order $\to$ Enjoy. Skip button available for returning users.
    \item \textbf{Consistent Interactions:} Once a user learns one pattern (e.g., tapping a card), it applies everywhere. All back buttons, toggles, and action buttons behave identically across screens.
\end{itemize}

% ===================================================================
\section{Mental Models}

Mental models are internal representations of how users expect systems to work.

\begin{itemize}
    \item \textbf{Restaurant Menu Model:} The restaurant page is organized by categories (Starters, Main Course, Desserts, Beverages) with item name, description, and price --- mimicking a real menu.
    \item \textbf{E-Commerce Cart Model:} Standard flow: add items $\to$ view cart $\to$ review $\to$ apply coupon $\to$ see total $\to$ place order. Matches the universal online shopping mental model.
    \item \textbf{Delivery Tracking Model:} Map-based tracking with a moving rider icon, ETA, delivery partner details, and contact options --- mirrors Uber/Google Maps experience.
\end{itemize}

% ===================================================================
\section{Closure}

Closure refers to the user's need to experience a clear sense of completion after an action.

\begin{itemize}
    \item \textbf{Order Confirmation Screen:} After placing an order, users see a large green checkmark, ``Order Placed!'' heading, confetti animation, order details (ID, restaurant, ETA, total), and ``Track Your Order'' / ``Back to Home'' buttons.
    \item \textbf{Tracking Steps:} Each step transitions from gray (upcoming) to green checkmark (completed). The progress line fills as steps complete.
    \item \textbf{Micro-closure:} Coupon applied $\to$ green success message. Item added $\to$ badge animation. OTP sent $\to$ timer revealed. Login complete $\to$ transitions to Home.
\end{itemize}

% ===================================================================
\newpage
\section{Inverted Pyramid}

The most important information appears first, with details in decreasing order of importance.

\subsection{Home Screen Hierarchy}
\begin{enumerate}
    \item Location and search bar (critical context and most-used action)
    \item Promotional banners (time-sensitive, high impact)
    \item Food categories (primary browsing method)
    \item Restaurant cards (exploration content)
    \item Popular picks (supporting discovery)
\end{enumerate}

\subsection{Restaurant Card Hierarchy}
\begin{enumerate}
    \item Restaurant name (identification)
    \item Rating + delivery time (decision factors)
    \item Cuisine type (categorization)
    \item Location + price (supporting detail)
\end{enumerate}
Users can make a go/no-go decision without reading past the second line.

% ===================================================================
\section{Flexibility and Robustness}

\subsection{Multiple Paths to the Same Goal}
\begin{itemize}
    \item Browse: Home $\to$ Categories $\to$ Restaurant $\to$ Menu
    \item Search: Direct text search for dishes or restaurants
    \item Discover: Trending searches, popular near you, recommendations
    \item Reorder: One-tap reorder from order history
\end{itemize}

\subsection{Robustness}
\begin{itemize}
    \item Multiple login methods (Phone OTP, Google, Apple)
    \item Saved addresses and payment methods reduce friction
    \item Veg/Non-Veg filter for dietary preferences
    \item Multiple sort and filter options
    \item Contact options for both restaurant and delivery partner
\end{itemize}

% ===================================================================
\section{Asimov's Laws of Robotics Applied to UI Design}

Asimov's Three Laws of Robotics, when adapted to HCI, provide a framework for designing systems that prioritize user safety, obedience to intent, and self-preservation of state.

\subsection{First Law: Do Not Harm the User}
The system must never cause harm (data loss, confusion, wasted time, or frustration) to the user.
\begin{itemize}
    \item \textbf{No data loss:} Cart contents persist across screen transitions. Users never lose items they have added.
    \item \textbf{Clear error prevention:} Phone validation prevents submission of invalid data. OTP auto-advance reduces misentry.
    \item \textbf{Transparent pricing:} Full bill breakdown (item total, delivery fee, taxes, discount) shown before checkout --- no hidden charges.
    \item \textbf{Confirmation before irreversible actions:} Order placement requires explicit ``Place Order'' tap with full summary visible.
    \item \textbf{Safe defaults:} Default delivery address (Home) and default item size (Regular) prevent accidental wrong orders.
\end{itemize}

\subsection{Second Law: Obey the User's Intent}
The system must faithfully execute what the user wants, unless doing so would violate the First Law.
\begin{itemize}
    \item \textbf{Direct response:} Tapping ``Add'' immediately adds the item to cart with visual confirmation (badge update).
    \item \textbf{Respect preferences:} Veg/Non-Veg toggle filters the menu and persists the user's dietary preference.
    \item \textbf{Flexible navigation:} Back buttons, bottom nav, and skip buttons all honor the user's navigational intent.
    \item \textbf{Coupon application:} When user applies a coupon, the discount is immediately reflected in the bill.
    \item \textbf{Cancel order:} The tracking screen provides a cancel button --- honoring the user's change of mind.
\end{itemize}

\subsection{Third Law: Preserve System State}
The system must protect its own consistent state, unless doing so conflicts with the First or Second Law.
\begin{itemize}
    \item \textbf{State persistence:} Cart state, selected address, and login status are maintained across navigation.
    \item \textbf{Consistent UI state:} Bottom navigation always reflects the current active screen. Cart badge always shows the correct count.
    \item \textbf{Graceful transitions:} Screen transitions use smooth animations --- the UI never enters a broken or intermediate visual state.
    \item \textbf{Order tracking integrity:} The tracking progress only moves forward (Placed $\to$ Preparing $\to$ Out $\to$ Delivered), maintaining logical consistency.
\end{itemize}

% ===================================================================
\section{Balancing Tesler's Law and the Vital Few (80-20 Rule)}

The assignment emphasizes that the design should strike the right balance between Tesler's Law (conservation of complexity) and the Vital Few / 80-20 Rule. Here is how QuickBite achieves this balance:

\subsection{Where Tesler's Law Dominates (System Absorbs Complexity)}
\begin{itemize}
    \item \textbf{Bill computation:} The system auto-calculates item totals, delivery fees, taxes, discounts, and grand total. The user sees only the final result.
    \item \textbf{OTP auto-advance:} After entering a digit, the cursor automatically moves to the next box --- the system handles focus management.
    \item \textbf{Address selection:} Previously saved addresses (Home, Work) appear as one-tap options rather than requiring re-entry.
    \item \textbf{Smart defaults:} Regular size and default address are pre-selected, so the user only acts when they want to change something.
\end{itemize}

\subsection{Where the 80-20 Rule Dominates (Focus on Critical Features)}
\begin{itemize}
    \item \textbf{Core 20\%:} Browse $\to$ Search $\to$ Restaurant $\to$ Cart $\to$ Order $\to$ Track. These screens receive maximum design attention, occupy the most screen real estate, and are reachable in 1--2 taps.
    \item \textbf{Supporting 80\%:} Profile, Help, Notifications, Order History are organized in bottom nav and accessible but never dominate the primary flow.
\end{itemize}

\subsection{The Balance}
\begin{itemize}
    \item For the \textbf{critical 20\%} features, Tesler's Law is applied aggressively --- the system absorbs as much complexity as possible (auto-calculation, smart defaults, auto-advance).
    \item For the \textbf{supporting 80\%} features, minimal complexity is introduced --- simple list views, standard patterns, and secondary placement keep them out of the user's way.
    \item The result: the most-used features are both \textbf{prominent} (80-20) and \textbf{effortless} (Tesler's), while secondary features remain accessible without adding cognitive load.
\end{itemize}

% ===================================================================
\newpage
\section{Screen-by-Screen Design with Screenshots}

Each screen is shown with its screenshot and the key HCI concepts applied.

% --- SPLASH ---
\subsection{Screen 1: Splash Screen}

\begin{minipage}[t]{0.42\textwidth}
\vspace{0pt}
\begin{center}
\fbox{\includegraphics[width=\linewidth]{screenshots/01_splash.png}}
\captionof{figure}{Splash Screen}
\end{center}
\end{minipage}
\hfill
\begin{minipage}[t]{0.54\textwidth}
\vspace{0pt}
\textbf{HCI Concepts:}
\begin{itemize}[noitemsep, leftmargin=*]
    \item Visibility of System Status (Nielsen H1): Loading animation provides feedback
    \item Aesthetic and Minimalist Design (Nielsen H8): Clean gradient with logo
    \item Auto-transitions to onboarding after 2.5 seconds
\end{itemize}
\end{minipage}

\vspace{16pt}

% --- ONBOARDING ---
\subsection{Screen 2: Onboarding (3 Slides)}

\begin{figure}[H]
\centering
\begin{subfigure}[b]{0.30\textwidth}
    \fbox{\includegraphics[width=\linewidth]{screenshots/02_onboarding_1.png}}
    \caption{Browse}
\end{subfigure}
\hfill
\begin{subfigure}[b]{0.30\textwidth}
    \fbox{\includegraphics[width=\linewidth]{screenshots/02_onboarding_2.png}}
    \caption{Order}
\end{subfigure}
\hfill
\begin{subfigure}[b]{0.30\textwidth}
    \fbox{\includegraphics[width=\linewidth]{screenshots/02_onboarding_3.png}}
    \caption{Enjoy}
\end{subfigure}
\caption{Onboarding Screens}
\end{figure}

\begin{itemize}[noitemsep]
    \item Learnability: Introduces app in 3 simple steps
    \item User Control (Nielsen H3): Skip button for returning users
    \item System Status (Nielsen H1): Progress dots show current position
    \item Closure: ``Get Started'' on final slide
    \item Serial Position: Most memorable content at start and end
\end{itemize}

\newpage

% --- LOGIN ---
\subsection{Screen 3: Login / Sign Up}

\begin{minipage}[t]{0.42\textwidth}
\vspace{0pt}
\begin{center}
\fbox{\includegraphics[width=\linewidth]{screenshots/03_login.png}}
\captionof{figure}{Login Screen}
\end{center}
\end{minipage}
\hfill
\begin{minipage}[t]{0.54\textwidth}
\vspace{0pt}
\textbf{HCI Concepts:}
\begin{itemize}[noitemsep, leftmargin=*]
    \item Error Prevention (Nielsen H5): Real-time phone validation
    \item Error Recovery (Nielsen H9): Specific error messages
    \item Flexibility (Nielsen H7): Phone OTP, Google, and Apple sign-in
    \item Feedback (Shneiderman R3): OTP auto-advance, focus states
    \item Consistency (Shneiderman R1): Same button/form patterns
\end{itemize}
\end{minipage}

\vspace{16pt}

% --- HOME ---
\subsection{Screen 4: Home Dashboard}

\begin{minipage}[t]{0.42\textwidth}
\vspace{0pt}
\begin{center}
\fbox{\includegraphics[width=\linewidth]{screenshots/04_home.png}}
\captionof{figure}{Home Screen}
\end{center}
\end{minipage}
\hfill
\begin{minipage}[t]{0.54\textwidth}
\vspace{0pt}
\textbf{HCI Concepts:}
\begin{itemize}[noitemsep, leftmargin=*]
    \item Inverted Pyramid: Location + search at top, then banners, categories, restaurants
    \item 80-20 Rule: Browse and search dominate the screen
    \item Recognition over Recall (Nielsen H6): Emoji categories, visual cards
    \item Real-world Match (Nielsen H2): Natural language, food imagery
    \item Memory Load (Shneiderman R8): Cart badge, notification badge
    \item Serial Position: Home is first in bottom nav
\end{itemize}
\end{minipage}

\vspace{16pt}

% --- SEARCH ---
\subsection{Screen 5: Search Screen}

\begin{minipage}[t]{0.42\textwidth}
\vspace{0pt}
\begin{center}
\fbox{\includegraphics[width=\linewidth]{screenshots/05_search.png}}
\captionof{figure}{Search Screen}
\end{center}
\end{minipage}
\hfill
\begin{minipage}[t]{0.54\textwidth}
\vspace{0pt}
\textbf{HCI Concepts:}
\begin{itemize}[noitemsep, leftmargin=*]
    \item Recognition over Recall (Nielsen H6): Recent searches, trending items
    \item Flexibility (Nielsen H7): Filter pills, sort options, cuisine filters
    \item User Control (Shneiderman R7): Multiple filter/sort combinations
    \item Reversal (Shneiderman R6): Clear recent searches
    \item Consistency: Same pill/chip styling across app
\end{itemize}
\end{minipage}

\newpage

% --- RESTAURANT ---
\subsection{Screen 6: Restaurant Listing and Menu}

\begin{minipage}[t]{0.42\textwidth}
\vspace{0pt}
\begin{center}
\fbox{\includegraphics[width=\linewidth]{screenshots/06_restaurant.png}}
\captionof{figure}{Restaurant Screen}
\end{center}
\end{minipage}
\hfill
\begin{minipage}[t]{0.54\textwidth}
\vspace{0pt}
\textbf{HCI Concepts:}
\begin{itemize}[noitemsep, leftmargin=*]
    \item Mental Model: Menu organized like a real restaurant
    \item Real-world Match (Nielsen H2): Veg/Non-Veg standard symbols
    \item Memory Load (Shneiderman R8): Floating cart bar shows count + total
    \item Tesler's Law: Customization in popup, not inline
    \item User Control: Veg/Non-Veg toggle, category pills, Add buttons
    \item Inverted Pyramid: Restaurant info first, then menu
\end{itemize}
\end{minipage}

\vspace{12pt}

\begin{minipage}[t]{0.42\textwidth}
\vspace{0pt}
\begin{center}
\fbox{\includegraphics[width=\linewidth]{screenshots/06b_customize.png}}
\captionof{figure}{Item Customization Popup}
\end{center}
\end{minipage}
\hfill
\begin{minipage}[t]{0.54\textwidth}
\vspace{0pt}
\textbf{HCI Concepts:}
\begin{itemize}[noitemsep, leftmargin=*]
    \item Tesler's Law: Complex customization contained in bottom sheet
    \item Smart Defaults: Regular size pre-selected
    \item User Control: Size (radio), toppings (checkboxes), special instructions
    \item Freedom (Nielsen H3): Close button, quantity controls
    \item Closure: ``Add Item'' button confirms the action
\end{itemize}
\end{minipage}

\vspace{16pt}

% --- CART ---
\subsection{Screen 7: Cart and Checkout}

\begin{minipage}[t]{0.42\textwidth}
\vspace{0pt}
\begin{center}
\fbox{\includegraphics[width=\linewidth]{screenshots/07_cart.png}}
\captionof{figure}{Cart Screen}
\end{center}
\end{minipage}
\hfill
\begin{minipage}[t]{0.54\textwidth}
\vspace{0pt}
\textbf{HCI Concepts:}
\begin{itemize}[noitemsep, leftmargin=*]
    \item Tesler's Law: Bill auto-calculated by the system
    \item Reversal (Shneiderman R6): Quantity +/- controls, change address
    \item Feedback (Shneiderman R3): Coupon success message
    \item Error Prevention (Nielsen H5): Full breakdown before checkout
    \item Closure (Shneiderman R4): Prominent ``Place Order'' button
    \item Inverted Pyramid: Items, then billing, then action
\end{itemize}
\end{minipage}

\newpage

% --- CONFIRMATION ---
\subsection{Screen 8: Order Confirmation}

\begin{minipage}[t]{0.42\textwidth}
\vspace{0pt}
\begin{center}
\fbox{\includegraphics[width=\linewidth]{screenshots/08_confirmation.png}}
\captionof{figure}{Order Confirmation}
\end{center}
\end{minipage}
\hfill
\begin{minipage}[t]{0.54\textwidth}
\vspace{0pt}
\textbf{HCI Concepts:}
\begin{itemize}[noitemsep, leftmargin=*]
    \item Closure: Green checkmark, confetti animation, ``Order Placed!''
    \item Feedback (Shneiderman R3): Order ID, restaurant, ETA, total
    \item Dialog Yield Closure (Shneiderman R4): Clear end of ordering sequence
    \item Next steps: ``Track Your Order'' and ``Back to Home'' buttons
\end{itemize}
\end{minipage}

\vspace{16pt}

% --- TRACKING ---
\subsection{Screen 9: Order Tracking}

\begin{minipage}[t]{0.42\textwidth}
\vspace{0pt}
\begin{center}
\fbox{\includegraphics[width=\linewidth]{screenshots/09_tracking.png}}
\captionof{figure}{Order Tracking}
\end{center}
\end{minipage}
\hfill
\begin{minipage}[t]{0.54\textwidth}
\vspace{0pt}
\textbf{HCI Concepts:}
\begin{itemize}[noitemsep, leftmargin=*]
    \item System Status (Nielsen H1): 4-step progress with ETA countdown
    \item Mental Model: Map tracking mirrors Uber/Google Maps
    \item Real-world Match (Nielsen H2): Rider icon, restaurant and home markers
    \item User Control (Shneiderman R7): Contact delivery partner, cancel order
    \item Closure: Steps complete with green checkmarks
\end{itemize}
\end{minipage}

\vspace{16pt}

% --- ORDERS ---
\subsection{Screen 10: Order History}

\begin{minipage}[t]{0.42\textwidth}
\vspace{0pt}
\begin{center}
\fbox{\includegraphics[width=\linewidth]{screenshots/10_orders.png}}
\captionof{figure}{Order History}
\end{center}
\end{minipage}
\hfill
\begin{minipage}[t]{0.54\textwidth}
\vspace{0pt}
\textbf{HCI Concepts:}
\begin{itemize}[noitemsep, leftmargin=*]
    \item Shortcuts (Shneiderman R2): ``Reorder'' button for repeat ordering
    \item Recognition (Nielsen H6): Order cards show restaurant, items, status, date
    \item Closure: Status badges (Delivered, Cancelled, Preparing)
    \item Consistency: Same card pattern as rest of app
    \item Flexibility: Tabs for Ongoing/Past orders; rate and review option
\end{itemize}
\end{minipage}

\newpage

% --- PROFILE ---
\subsection{Screen 11: Profile / Account}

\begin{minipage}[t]{0.42\textwidth}
\vspace{0pt}
\begin{center}
\fbox{\includegraphics[width=\linewidth]{screenshots/11_profile.png}}
\captionof{figure}{Profile Screen}
\end{center}
\end{minipage}
\hfill
\begin{minipage}[t]{0.54\textwidth}
\vspace{0pt}
\textbf{HCI Concepts:}
\begin{itemize}[noitemsep, leftmargin=*]
    \item 80-20 Rule: Secondary features in a clear menu hierarchy
    \item Shortcuts: Saved addresses, linked payments, wallet balance
    \item User Control: Notification toggle, edit profile, manage payments
    \item Consistency: Card-based sections, same icon styling
    \item Serial Position (Recency): Profile at last position in bottom nav
\end{itemize}
\end{minipage}

\vspace{16pt}

% --- HELP ---
\subsection{Screen 12: Help and Support}

\begin{minipage}[t]{0.42\textwidth}
\vspace{0pt}
\begin{center}
\fbox{\includegraphics[width=\linewidth]{screenshots/12_help.png}}
\captionof{figure}{Help and Support}
\end{center}
\end{minipage}
\hfill
\begin{minipage}[t]{0.54\textwidth}
\vspace{0pt}
\textbf{HCI Concepts:}
\begin{itemize}[noitemsep, leftmargin=*]
    \item Help and Documentation (Nielsen H10): FAQ accordion, live chat, call, email
    \item Error Recovery: Issue reporting form with order dropdown
    \item Flexibility: Three contact methods for different preferences
    \item Recognition: Visual icons for quick actions
    \item 80-20: Help is accessible but secondary
\end{itemize}
\end{minipage}

\vspace{16pt}

% --- NOTIFICATIONS ---
\subsection{Screen 13: Notifications}

\begin{minipage}[t]{0.42\textwidth}
\vspace{0pt}
\begin{center}
\fbox{\includegraphics[width=\linewidth]{screenshots/13_notifications.png}}
\captionof{figure}{Notifications}
\end{center}
\end{minipage}
\hfill
\begin{minipage}[t]{0.54\textwidth}
\vspace{0pt}
\textbf{HCI Concepts:}
\begin{itemize}[noitemsep, leftmargin=*]
    \item System Status (Nielsen H1): Order updates with timestamps
    \item Recognition: Visual distinction between read and unread
    \item Categorization: Distinct icons for order vs promotional notifications
    \item User Control: ``Mark all read'' button
    \item Consistency: Same list pattern and styling as rest of app
\end{itemize}
\end{minipage}

% ===================================================================
\newpage
\section{User Flow}

\subsection{Primary Flow}

\begin{center}
Splash $\to$ Onboarding $\to$ Login $\to$ \textbf{Home} $\to$ \textbf{Restaurant} $\to$ \textbf{Cart} $\to$ \textbf{Confirmation} $\to$ \textbf{Tracking}
\end{center}

\subsection{Alternative Flows}
\begin{itemize}
    \item \textbf{Search Flow:} Home $\to$ Search $\to$ Restaurant $\to$ Cart $\to$ Confirmation
    \item \textbf{Reorder Flow:} Orders $\to$ Reorder $\to$ Cart $\to$ Confirmation
    \item \textbf{Category Flow:} Home $\to$ Category $\to$ Restaurant $\to$ Cart $\to$ Confirmation
\end{itemize}

% ===================================================================
\newpage
\section{How to Test the Prototype}

The QuickBite prototype is a fully interactive HTML/CSS/JS application. Follow the steps below to run and test it.

\subsection{Prerequisites}
\begin{itemize}
    \item A modern web browser (Google Chrome, Firefox, or Microsoft Edge)
    \item No installation or server setup is required
\end{itemize}

\subsection{Running the Prototype}
\begin{enumerate}
    \item Clone or download the repository from GitHub:
    \begin{verbatim}
    git clone https://github.com/Niranjan7771/HCI_MID.git
    \end{verbatim}
    \item Open the folder \texttt{HCI\_MID/}
    \item Double-click \texttt{index.html} to open it in your default browser
    \item The app will start with the Splash Screen and auto-transition to Onboarding
\end{enumerate}

\subsection{Navigating the Screens}
\begin{itemize}
    \item \textbf{Onboarding:} Swipe through 3 slides using Next/dots, or click Skip
    \item \textbf{Login:} Enter any 10-digit phone number, then fill the OTP boxes
    \item \textbf{Home:} Browse categories, scroll through restaurants, tap banners
    \item \textbf{Search:} Type in the search bar, use filter/sort pills
    \item \textbf{Restaurant:} Browse menu, toggle Veg/Non-Veg, tap Add to add items, tap an item for customization popup
    \item \textbf{Cart:} Adjust quantities, apply coupon code \texttt{WELCOME50}, tap Place Order
    \item \textbf{Confirmation:} View order details, tap Track Your Order
    \item \textbf{Tracking:} See live progress steps, map, and delivery partner info
    \item \textbf{Orders / Profile / Help / Notifications:} Accessible via bottom navigation or header icons
\end{itemize}

\subsection{Keyboard Shortcuts (Quick Navigation)}
The prototype includes keyboard shortcuts for direct screen access during testing:

\begin{table}[H]
\centering
\small
\begin{tabular}{c l | c l}
\toprule
\textbf{Key} & \textbf{Screen} & \textbf{Key} & \textbf{Screen} \\
\midrule
1 & Splash & 6 & Restaurant \\
2 & Onboarding & 7 & Cart \\
3 & Login & 8 & Order Confirmation \\
4 & Home & 9 & Order Tracking \\
5 & Search & 0 & Orders / History \\
\bottomrule
\end{tabular}
\caption{Keyboard shortcuts for quick screen navigation}
\end{table}

A navigation guide panel is also available on the left side of the prototype for click-based direct access to any screen.

% ===================================================================
\section{Conclusion}

The QuickBite food delivery app demonstrates a comprehensive application of HCI principles:

\begin{itemize}
    \item \textbf{80-20 Rule:} Core ordering flow receives primary design attention
    \item \textbf{Shneiderman's 8 Rules:} Consistency, feedback, closure, error handling, reversal, memory load reduction
    \item \textbf{Nielsen's 10 Heuristics:} System visibility, real-world match, user control, aesthetics, help
    \item \textbf{Tesler's Law:} Computation and decisions offloaded to the system
    \item \textbf{Serial Position:} Critical actions at start/end of navigation
    \item \textbf{Mental Models:} Design matches restaurant, e-commerce, and delivery tracking expectations
    \item \textbf{Closure:} Confirmation screens and success messages at every critical point
    \item \textbf{Inverted Pyramid:} Most important content first on every screen
    \item \textbf{Learnability:} Familiar patterns, consistent interactions, onboarding
    \item \textbf{Flexibility and Robustness:} Multiple paths, auth methods, filters, contact options
    \item \textbf{Asimov's Laws:} User safety (no hidden charges, error prevention), obedience to intent (direct responses, cancel support), state preservation (persistent cart, consistent UI)
    \item \textbf{Tesler's vs Vital Few Balance:} Critical 20\% features are both prominent and effortless; supporting 80\% are accessible but non-intrusive
\end{itemize}

The interactive prototype (HTML/CSS/JS) provides a hands-on demonstration of all 12 screens with working navigation, animations, and interactive elements.

% ===================================================================
\newpage
\section{Additional Links}

The following links provide access to the interactive prototype and source code:

\begin{table}[H]
\centering
\begin{tabular}{l l}
\toprule
\textbf{Resource} & \textbf{Link} \\
\midrule
Live Demo & \url{https://quickbite-4h67.onrender.com} \\
GitHub Repository & \url{https://github.com/Niranjan7771/HCI_MID} \\
Prototype (index.html) & Clone repo and open \texttt{index.html} in browser \\
LaTeX Source & \texttt{report.tex} in the repository \\
\bottomrule
\end{tabular}
\caption{Project links and resources}
\end{table}

\subsection{How to Access the Prototype}
The easiest way is to visit the live demo: \url{https://quickbite-4h67.onrender.com}

Alternatively, to run locally:
\begin{enumerate}
    \item Visit \url{https://github.com/Niranjan7771/HCI_MID}
    \item Click \textbf{Code $\to$ Download ZIP} or run:
    \begin{verbatim}
    git clone https://github.com/Niranjan7771/HCI_MID.git
    \end{verbatim}
    \item Open \texttt{index.html} in any modern browser (Chrome, Firefox, Edge)
    \item Navigate through all 12 screens using the bottom navigation bar, on-screen buttons, or keyboard shortcuts (keys 1--0)
\end{enumerate}

\subsection{Files in the Repository}
\begin{itemize}
    \item \texttt{index.html} --- Interactive prototype with 12 screens
    \item \texttt{styles.css} --- Complete design system and styling
    \item \texttt{app.js} --- Navigation logic and interactivity
    \item \texttt{report.tex} --- LaTeX source for this report
    \item \texttt{report.pdf} --- This compiled report
    \item \texttt{screenshots/} --- All screen captures used in this report
    \item \texttt{README.md} --- Project overview and testing instructions
\end{itemize}

\end{document}
